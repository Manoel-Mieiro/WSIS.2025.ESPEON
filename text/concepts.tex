% \section{Conceitos Gerais}\label{sec:concepts}  

% \subsection{ESPEON}
% Engajamento e Supervisão do Processo de Ensino Online (ESPEON) é uma extensão para navegadores Chrome, em desenvolvimento para medir os níveis de atenção e engajamento de alunos em aulas remotas. Será utilizada para validar a efetividade da metodologia expositiva no contexto do ensino remoto.

% \subsection{Metodologia Expositiva}
% Andreata (2019) \cite{ANDREATA2019} define a metodologia expositiva como aquela baseada na transmissão do saber por parte de um orador, por meio da exposição do conteúdo, enquanto ao estudante cabe uma postura predominantemente passiva de recepção e assimilação do conhecimento. 

% \subsection{MongoDB}
% MongoDB é um banco de dados não relacional de propósito geral, orientado a documentos, amplamente utilizado no desenvolvimento de aplicações modernas e escaláveis, especialmente em ambientes de computação em nuvem.

% \subsection{AtlasDB}
% AtlasDB é um serviço de banco de dados na nuvem oferecido pela MongoDB Inc., que facilita a implantação, o gerenciamento e o escalonamento de instâncias MongoDB.

% \subsection{API}
% Sigla de \textit{Application Programming Interface}, uma API é um conjunto de definições e protocolos que permite a comunicação entre sistemas, aplicativos ou serviços.

% \subsection{Flask}
% Flask é um framework da linguagem Python voltado para aplicações web. Ele permite o desenvolvimento ágil de sistemas robustos, com suporte à escalabilidade e integração com diversas bibliotecas.

% \subsection{\textit{Deploy}}
% Termo em inglês comumente traduzido como "implantação", refere-se ao processo de disponibilização de um sistema ou aplicação em ambiente de produção.

% \subsection{\textit{Log}}
% Em computação, \textit{logs} são registros estruturados que documentam eventos ou ações executadas por um sistema, comumente utilizados para auditoria, monitoramento e análise de comportamento.

% % \subsection{\textit{Payload}}


% \subsection{Chrome Web Store}
% Plataforma oficial de distribuição de extensões, aplicações e temas para o navegador Google Chrome.


\section{Conceitos Gerais}\label{sec:concepts}
Este trabalho apresenta conceitos fundamentais relacionados ao desenvolvimento e funcionamento da extensão ESPEON (Engajamento e Supervisão do Processo de Ensino Online), uma ferramenta desenvolvida para navegadores Google Chrome com o objetivo de mensurar os níveis de atenção e engajamento de alunos em aulas remotas. A proposta da ESPEON é validar a efetividade da metodologia expositiva no contexto do ensino remoto, metodologia esta que, segundo Andreata (2019) \cite{ANDREATA2019}, baseia-se na transmissão de conteúdo por um orador, cabendo ao estudante uma postura predominantemente passiva de recepção e assimilação do conhecimento.

Para possibilitar o funcionamento da ESPEON, algumas tecnologias foram empregadas. Primeiro, o MongoDB - um banco de dados não relacional - orientado a documentos, amplamente utilizado no desenvolvimento de aplicações modernas, principalmente em ambientes que exigem escalabilidade, como os baseados em computação em nuvem. Esse banco de dados é gerenciado na nuvem por meio do serviço AtlasDB, oferecido pela MongoDB Inc., o qual simplifica a implantação, o gerenciamento e o escalonamento de instâncias MongoDB.

A comunicação entre a extensão e o MongoDB ocorre por meio de uma API (\textit{Application Programming Interface}), que define protocolos e regras para a interação entre sistemas, aplicativos ou serviços. Para este projeto, utiliza-se o framework Flask, escrito em Python, para a criação das APIs. Esse framework permite a criação ágil de aplicações web escaláveis e integráveis com diferentes bibliotecas.

Além disso, o processo de implantação da aplicação em ambiente de produção, o \textit{deploy}, ocorre, neste contexto, quando o banco de dados e a API são disponibilizados para acesso por inúmeros clientes. A operação e o comportamento da aplicação são constantemente monitorados por meio de \textit{logs}, que são registros estruturados de eventos e ações executadas pelo sistema. Por fim, vale destacar que a distribuição da extensão ocorre por meio da Chrome Web Store, a plataforma oficial do Google para disponibilização de extensões, aplicações e temas voltados ao navegador Chrome.