\section{Introdução}\label{sec:introdução}

A metodologia expositiva de ensino, iniciada com o patriarca da pedagogia, Platão (427–347 a.C.), é uma técnica tradicional, que perdura até os dias atuais \cite{ANDREATA2019}, apesar dos inúmeros avanços tecnológicos ocorridos ao longo do tempo. Seu elemento central é a oratória, por meio da qual o educador transmite seu saber aos alunos. Com isso, há uma hierarquia clara: o professor é o centro no processo de aprendizagem, enquanto o aluno é coadjuvante, sendo ele responsável pela retenção do conteúdo.

No entanto, ainda que tradicional, a metodologia deve ser revista, dado o presente contexto tecnológico. Autores como Andreata (2019) \cite{ANDREATA2019} e Pereira e Silva (2022) \cite{pereira2022critica} demonstram preocupação com a retenção do conteúdo pelo aluno; pois esse tem um papel passivo em seu próprio processo de aprendizado. Além disso, hoje, é preciso se preocupar também com os inúmeros estímulos externos advindos da exposição às telas \cite{lopes2017uso}. Essa realidade torna-se ainda mais preocupante em países com dificuldades de acesso à tecnologia, como o Brasil, onde cresce o número de estudantes no Ensino Superior em modalidades à distância (EAD) \cite{moran2009ensino}, sem que haja necessariamente um aumento no acompanhamento individualizado desses alunos, para garantir que estão aprendendo. 

Diante desse cenário, é fundamental repensar as estratégias de ensino no contexto das aulas remotas, de forma a tornar o processo de aprendizagem mais eficaz e menos exaustivo para todos os envolvidos.

Nesse sentido, este artigo propõe o uso da ferramenta ESPEON — uma extensão do Google Chrome — para monitoramento e análise do foco e engajamento de alunos durante aulas remotas. A ferramenta coleta, em tempo real, dados de navegação e o estado dos periféricos (câmera e microfone), os quais são armazenados em um banco de dados e, posteriormente, utilizados para gerar relatórios com métricas relevantes sobre a atenção e o engajamento dos estudantes. Essas informações auxiliam o docente na avaliação da qualidade do ensino remoto e na identificação de oportunidades de melhoria.
