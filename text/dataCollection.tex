\section{Coleta e Análise de Dados}\label{sec:datacollection}
A coleta dos dados será realizada em ambiente remoto de ensino, com turmas distintas, visando à obtenção de resultados mais plurais. Pretende-se conduzir o estudo em três disciplinas de áreas do conhecimento distintas, com o objetivo de observar variações nos níveis de atenção e engajamento, por análise quantitativa dos dados extraídos do ESPEON, permitindo uma análise mais precisa mais precisa acerca da efetividade da aula remota expositiva.

O estudo será iniciado com o docente criando uma aula na aplicação, disponibilizando o link de acesso aos alunos; que farão fazer sua inserção na ferramenta para inicialização da gravação durante todo o período da aula. Ao término da sessão, a aplicação gerará automaticamente um relatório com os dados coletados, o qual será enviado em formato PDF ao docente responsável, por meio de notificação por \textit{e-mail}.

Os relatórios quantitativos obtidos ao longo dos experimentos serão analisados com base em variáveis como número de participantes, duração das aulas e área do conhecimento. A análise estatística descritiva será utilizada para identificar padrões e tendências, permitindo construir conclusões sobre a efetividade da metodologia expositiva aplicada ao ensino remoto.