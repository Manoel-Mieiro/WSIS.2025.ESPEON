\subsection{Métodos}\label{sec:methods}
Esta pesquisa adota uma abordagem quantitativa, descritiva-experimental, cujo objetivo é mensurar os níveis de atenção e engajamento de estudantes em aulas remotas. A coleta de dados será feita por meio da extensão ESPEON, aplicada em turmas de diferentes áreas do conhecimento, a fim de garantir maior representatividade e diversidade no conjunto de dados.

O experimento será conduzido com a participação voluntária de alunos matriculados em três disciplinas distintas, no contexto do ensino superior; com amostragem não probabilística, por conveniência. Os dados serão coletados a partir dos \textit{logs} de navegação gerados pela extensão, incluindo eventos como mudanças de aba, minimização da janela da conferência, e o uso de periféricos como microfone e câmera.

Os indicadores extraídos serão armazenados em um banco de dados não relacional, MongoDB, e processados no \textit{backend} Python da aplicação, para posteriormente gerar relatórios de desempenho em formato PDF. A análise dos dados será realizada por meio de estatísticas descritivas — como média, desvio padrão e frequências —  com o objetivo de identificar padrões de comportamento entre os grupos analisados. A comparação entre disciplinas permitirá avaliar o impacto da área do conhecimento sobre os níveis de atenção e engajamento dos estudantes.

Todos os participantes serão informados previamente sobre os objetivos do estudo, e o uso dos dados seguirá as diretrizes éticas da pesquisa com seres humanos, com ênfase na privacidade e na conformidade com a Lei Geral de Proteção de Dados (LGPD).
