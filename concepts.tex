\section{Conceitos Gerais}\label{sec:related}

Diversas pesquisas vêm sendo desenvolvidas para verificar o nível de atenção dos alunos em ambientes de ensino remoto, sendo especialmente impulsionadas pelo contexto da pandemia de COVID-19, que impôs o Ensino a Distância (EAD) como medida emergencial, que perdura até a atualidade. Nesse cenário, \cite{ong2021application} propôs a detecção de atenção com base na captura de imagens dos alunos. Para distinguir entre alunos focados e dispersos, foram utilizados os algoritmos Viola-Jones e Sobel Edge, com o objetivo de identificar componentes faciais (como rosto, boca e olhos) e detectar o estado dos olhos (abertos ou fechados), respectivamente.

\cite{ong2021application} justifica o uso de imagens — um dado extremamente sensível — com base em dois grandes desafios enfrentados na detecção de atenção. O primeiro refere-se à alternativa de solicitar que os alunos interajam respondendo a perguntas; no entanto, essa abordagem pode ser ineficaz, já que nem todos respondem, e, quando o fazem, podem apresentar lentidão, atrasando o andamento da aula. A segunda alternativa seria solicitar que os alunos mantenham a câmera ligada; porém, além de exigir a verificação manual por parte do professor, os alunos ainda poderiam utilizar outros dispositivos, relegando a aula a segundo plano.

Embora o estudo tenha obtido sucesso na identificação dos elementos faciais, apresentou falhas na detecção consistente do estado dos olhos ao utilizar o algoritmo Sobel Edge.

Outros trabalhos, como o de \cite{saha2021attendance}, constataram — por meio de análise qualitativa — que 80\% dos alunos entrevistados admitiam apresentar déficit de atenção durante as aulas remotas, enquanto 40\% declaravam comparecer apenas para garantir a presença, sem engajamento ou atenção efetiva. Diante disso, esta pesquisa propõe a implementação de um software monitor que utiliza dados de imagem do usuário combinados com modelos de inteligência artificial (IA) para avaliar o grau de atenção do indivíduo e, com base nisso, atribuir a presença. O programa monitoraria a conectividade de rede dos alunos, ativaria as câmeras e, em intervalos regulares, capturaria imagens dos usuários. De posse dessas imagens, o sistema verificaria por quanto tempo o aluno esteve olhando para fora da tela, processando-as localmente no cliente e comparando os resultados a modelos treinados com IA. Por fim, os professores apenas consumiriam o resultado apontado pelo sistema, referente ao nível de atenção de cada aluno.

Diferentemente desses trabalhos, o presente estudo propõe uma abordagem menos invasiva e mais centrada na experiência do aluno. Por meio da coleta de dados de navegação, Engajamento e Supervisão do Processo Educacional Online (ESPEON) busca identificar padrões de atenção e engajamento sem recorrer a imagens ou gravações. Esses indicadores servirão como base para uma análise crítica da eficácia da metodologia expositiva em ambientes remotos de ensino, respeitando os princípios de privacidade e a conformidade com a Lei Geral de Proteção de Dados (LGPD).
